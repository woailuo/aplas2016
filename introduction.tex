\section{Introduction}
\label{sec:introduction}
Manual memory mangagement primitives (e.g. \(\texttt{malloc}\) and
\(\texttt{free}\) in C language) are a very flexible way to manage
computer memory cells.  We can write a program which dynamically
allocates a memory cell during running and deallocates a memory cell
when it is no longer used. However, manual memory management
primitives often cause hard-to-find problems, for example, double
frees (\texttt{free} a deallocated memory cell ), memory leaks (forget
to deallocate memory cells) and illegal accesses to a dangling
pointer. Therefore, many static verification methods have been
proposed to guarantee safe memory deallocation. They prove
\emph{partial} memory-leak freedom: if a program terminates, all the
memory cells are safe deallocated. As we know that nonterminating
programs are very common in real-world programmings such as Web
servers and operating systems. To guarantee \emph{total} memory-leak
freedom, if a program does not consume unbounded number of memory
cells during execution, is a very crucial issue.
